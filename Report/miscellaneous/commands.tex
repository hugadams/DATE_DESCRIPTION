%%%%%%%%%%%%%%%%%%%%%%%%%%%%
%%%% CUSTOM TEX COMMANDS %%%
%%%%%%%%%%%%%%%%%%%%%%%%%%%%    
    
%%% Special colors/fonts
\newcommand{\curious}[1]{
   {\bf \color{green}#1}
   }        
        
\newcommand{\good}[1]{
   {\bf\color{blue}#1}
   }

\newcommand{\bad}[1]{
   {\bf\color{red}#1}
   }

\newcommand{\blf}[1]{
   %Bold, large, non-indented line
   {\bf \noindent \large #1}
   }   

\newcommand{\conc}[1]{
   %Bold, purple --> Used to make concentration estimates pop
   {\bf \color{purple}#1}
   }  

\newcommand{\fillin}[1]{  
   % Orange bold text for sections that need filled in
   {\bf \ttfamily \color{orange} .... \textsc{#1} ....}
   }
   
\newcommand{\todo}[1]{
   % Large, eye-grabbing "to do" notes; used for followup
   {\noindent \bf \large \ttfamily \color{red}::::::::::::::::::::::
    {{\color{black} \\ #1} \\ :::::::::::::::::::::: }}
   }

   
%%% Molecules %%%%
\newcommand*{\water}{H$_2$O } % Need these spaces or pushed up on text
\newcommand*{\ethanol}{C$_2$H$_6$O }
\newcommand*{\peroxide}{H$_2$O$_2$ }
\newcommand*{\sulfuric}{H$_2$SO$_4$ }


%%% Lists/Spacing %%%
\newcommand*{\bit}{\begin{itemize}}
\newcommand*{\eit}{\end{itemize}}

\newcommand{\sub}[1]{
    % A list of one element: a bullet point
    \begin{itemize}
       \item{#1}
    \end{itemize}
    }

\newcommand*{\gap}{
	% Very common vertical space
    \vspace{.2cm}
    }


%%% Miscellanous %%%  
\newcommand{\texfromfile}[3]{
    % Tries to import a tex file; if not found, prints large error message   
    % To fit \subimport requirements, #1 must be directorypath and #2 must be filename
    % #3 is an optional message that is printed if file is missing (to give more info/directions)
    % THIS WILL ERROR IF FILE NAMES HAVE UNDERSCORES WHEN IT TRIES TO PRINT FILENAME!!!
    % hacky soln: http://tex.stackexchange.com/questions/38536/how-can-i-pass-underscore-to-newcommand-properly
    \IfFileExists{#1#2}
        {\subimport{#1}{#2}}  
    % Else
     { {\color{red} \huge \noindent Cannot find file #1#2} \ifthenelse{\isempty{#3}}{}{{\\ \bf \noindent #3}}} 
     
    }     
    
\newcommand{\blfootnote}[1]{
    % Footnote without a marker
    \begingroup
    \renewcommand\thefootnote{}\footnote{#1}%
    \addtocounter{footnote}{-1}%
    \endgroup
    }      

\newcommand{\blockquote}[1]{
    % Indents and changes font of a large block of text.  
    % Similar to blockquotes in MLA formatting
    {\begin{addmargin}[3em]{2em}\vspace{.2cm} \ttfamily #1 \vspace{.2cm} \end{addmargin}}
    }     
    
\newcommand{\fancyname}[2]{
   %Takes first and last name, changes font/capitalization}
   #1 \textsc{#2}
   }

\newcommand{\define}[2]{
   % Used for special formatting of terms/acronyms in appendix
   {\item{\small {\bf #1}:\hspace{.75cm} #2}}
   }

%%%%%%%%%%%%%%%%%%%%%%%%%%%%%%%%%%%%%%%%%%%%%%%%%%%
%%% Hyperlinking websites and local files %%%%%%%%%
%%%%%%%%%%%%%%%%%%%%%%%%%%%%%%%%%%%%%%%%%%%%%%%%%%%

\newcommand{\link}[2]{
  %Link to a url
  %#1 URL
  %#2 Text description
  \item{
     \href
        {#1}
        {\ttfamily \color{blue} #2}

     }
  }

\newcommand{\report}[2]{
    % Link to another file (usually .pdf) rooted in FiberData.
    % #1: Path relative to FiberData (ie October/10_15_DTSSP/report/report.pdf}
    % #2: Link name (if using underscores, pass \_ through.}
    \item{
        \href
             {run:\FiberData #1}
             {\ttfamily \color{blue} #2} 
         }
    }

\newcommand{\paper}[2]{
    % Link to another file (usually .pdf) rooted in FiberData.
    % #1: Path relative to FiberData (ie October/10_15_DTSSP/report/report.pdf}
    % #2: Link name (if using underscores, pass \_ through.}
    \item{
        \href
             {run:\Papers #1}
             {\ttfamily \color{blue} #2} 
        }
    }
        

\newcommand{\refer}[1]{
   % Shorthand for "Fig. (\ref{})"
   {Fig. (\ref{#1})}
   }  

% \makeatletter seems necessary}
\makeatletter
\newcommand{\hdir}[2]{
   % Open a folder (update evince)
    \hyper@linkurl{#2}{#1}
    }
\makeatother

