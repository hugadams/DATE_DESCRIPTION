\documentclass{article}

%%% IMPORTS %%%
\usepackage{amsmath}
\usepackage{amssymb}
\usepackage{amsthm}
\usepackage{import}
\usepackage{changepage} %For forcing table adjustmetns
\usepackage{gensymb}
\usepackage{xcolor}
\usepackage[pdftex]{graphicx}
\usepackage{scrextend}
\usepackage{pdfpages}
\graphicspath{ {./images/} }
\usepackage{subfigure}
\usepackage[scriptsize]{caption}   % Set figure captions
\usepackage{changepage} 
\usepackage{multicol}   %May be deprecated
\usepackage{enumitem}
\usepackage{xifthen}  %if/else for optional args
\usepackage{geometry}
\usepackage{hyperref} % For links in appendix
\hypersetup{
    colorlinks=true, %This is best setup for not getting werid behavior
    linkcolor=black,
    citecolor=black,
    filecolor=black,
    urlcolor=blue, 
    }

%%% INPUT SEVERAL COMMANDS FROM miscellaneous/commands.tex
%%%%%%%%%%%%%%%%%%%%%%%%%%%%
%%%% CUSTOM TEX COMMANDS %%%
%%%%%%%%%%%%%%%%%%%%%%%%%%%%    
    
%%% Special colors/fonts
\newcommand{\curious}[1]{
   {\bf \color{green}#1}
   }        
        
\newcommand{\good}[1]{
   {\bf\color{blue}#1}
   }

\newcommand{\bad}[1]{
   {\bf\color{red}#1}
   }

\newcommand{\blf}[1]{
   %Bold, large, non-indented line
   {\bf \noindent \large #1}
   }   

\newcommand{\conc}[1]{
   %Bold, purple --> Used to make concentration estimates pop
   {\bf \color{purple}#1}
   }  

\newcommand{\fillin}[1]{  
   % Orange bold text for sections that need filled in
   {\bf \ttfamily \color{orange} .... \textsc{#1} ....}
   }
   
\newcommand{\todo}[1]{
   % Large, eye-grabbing "to do" notes; used for followup
   {\noindent \bf \large \ttfamily \color{red}::::::::::::::::::::::
    {{\color{black} \\ #1} \\ :::::::::::::::::::::: }}
   }

   
%%% Molecules %%%%
\newcommand*{\water}{H$_2$O } % Need these spaces or pushed up on text
\newcommand*{\ethanol}{C$_2$H$_6$O }
\newcommand*{\peroxide}{H$_2$O$_2$ }
\newcommand*{\sulfuric}{H$_2$SO$_4$ }


%%% Lists/Spacing %%%
\newcommand*{\bit}{\begin{itemize}}
\newcommand*{\eit}{\end{itemize}}

\newcommand{\sub}[1]{
    % A list of one element: a bullet point
    \begin{itemize}
       \item{#1}
    \end{itemize}
    }

\newcommand*{\gap}{
	% Very common vertical space
    \vspace{.2cm}
    }


%%% Miscellanous %%%  
\newcommand{\texfromfile}[3]{
    % Tries to import a tex file; if not found, prints large error message   
    % To fit \subimport requirements, #1 must be directorypath and #2 must be filename
    % #3 is an optional message that is printed if file is missing (to give more info/directions)
    % THIS WILL ERROR IF FILE NAMES HAVE UNDERSCORES WHEN IT TRIES TO PRINT FILENAME!!!
    % hacky soln: http://tex.stackexchange.com/questions/38536/how-can-i-pass-underscore-to-newcommand-properly
    \IfFileExists{#1#2}
        {\subimport{#1}{#2}}  
    % Else
     { {\color{red} \huge \noindent Cannot find file #1#2} \ifthenelse{\isempty{#3}}{}{{\\ \bf \noindent #3}}} 
     
    }     
    
\newcommand{\blfootnote}[1]{
    % Footnote without a marker
    \begingroup
    \renewcommand\thefootnote{}\footnote{#1}%
    \addtocounter{footnote}{-1}%
    \endgroup
    }      

\newcommand{\blockquote}[1]{
    % Indents and changes font of a large block of text.  
    % Similar to blockquotes in MLA formatting
    {\begin{addmargin}[3em]{2em}\vspace{.2cm} \ttfamily #1 \vspace{.2cm} \end{addmargin}}
    }     
    
\newcommand{\fancyname}[2]{
   %Takes first and last name, changes font/capitalization}
   #1 \textsc{#2}
   }

\newcommand{\define}[2]{
   % Used for special formatting of terms/acronyms in appendix
   {\item{\small {\bf #1}:\hspace{.75cm} #2}}
   }

%%%%%%%%%%%%%%%%%%%%%%%%%%%%%%%%%%%%%%%%%%%%%%%%%%%
%%% Hyperlinking websites and local files %%%%%%%%%
%%%%%%%%%%%%%%%%%%%%%%%%%%%%%%%%%%%%%%%%%%%%%%%%%%%

\newcommand{\link}[2]{
  %Link to a url
  %#1 URL
  %#2 Text description
  \item{
     \href
        {#1}
        {\ttfamily \color{blue} #2}

     }
  }

\newcommand{\report}[2]{
    % Link to another file (usually .pdf) rooted in FiberData.
    % #1: Path relative to FiberData (ie October/10_15_DTSSP/report/report.pdf}
    % #2: Link name (if using underscores, pass \_ through.}
    \item{
        \href
             {run:\FiberData #1}
             {\ttfamily \color{blue} #2} 
         }
    }

\newcommand{\paper}[2]{
    % Link to another file (usually .pdf) rooted in FiberData.
    % #1: Path relative to FiberData (ie October/10_15_DTSSP/report/report.pdf}
    % #2: Link name (if using underscores, pass \_ through.}
    \item{
        \href
             {run:\Papers #1}
             {\ttfamily \color{blue} #2} 
        }
    }
        

\newcommand{\refer}[1]{
   % Shorthand for "Fig. (\ref{})"
   {Fig. (\ref{#1})}
   }  

% \makeatletter seems necessary}
\makeatletter
\newcommand{\hdir}[2]{
   % Open a folder (update evince)
    \hyper@linkurl{#2}{#1}
    }
\makeatother



%%% Relative paths for composite document integration %%%
\newcommand*{\Analysis}{../Analysis/}
\newcommand*{\SEM}{../SEM/}
\newcommand*{\Datafiles}{../Datafiles/}
\newcommand*{\FiberData}{../../../}
\newcommand*{\Papers}{../../../../Papers/}

%%% Various hyperlinks to targets distributed in document
\newcommand*{\refsem}{\hyperlink{sem}{\color{blue} \ttfamily Introduction}} 
\newcommand*{\refexp}{\hyperlink{exp}{\color{blue} \ttfamily Experimental}} 
\newcommand*{\refintro}{\hyperlink{intro}{\color{blue} \ttfamily SEM}} 
\newcommand*{\refspec}{\hyperlink{spec}{\color{blue} \ttfamily Spectrometer Protocol}} 
\newcommand*{\refresults}{\hyperlink{results}{\color{blue} \ttfamily Results}} 
\newcommand*{\refrelated}{\hyperlink{related}{\color{blue} \ttfamily Related Reports}} 

% COVERAGE TABLE
\newcommand*{\refcovtable}{\hyperlink{covtable}{\color{blue} \ttfamily Coverage Table}} 

%%%%%%%%%%%%%%%%%%%%%%%%
%%%% BEGIN DOCUMENT  %%%
%%%%%%%%%%%%%%%%%%%%%%%%    
      
\begin{document}
\noindent

\begin{center}

\includegraphics[width=0.4\textwidth]{gwulogo}~\\[1cm]

\textsc{\LARGE The George Washington University\vspace{.2cm}
Physics Department}\\[.4cm]

{\large \today}
\vspace{.5cm}

\hrule \vspace{.4cm}
{ \huge \bfseries  

%%% TITLE %%%%
UNTITLED

\vspace{0.2cm} }
\hrule \vspace{1.5cm}

\begin{minipage}{0.4\textwidth}
\begin{flushleft} \large
\emph{Author:}\\

%%%% AUTHOR NAME %%%%%
\fancyname{Adam}{Hughes}\footnotemark[1]

\end{flushleft}
\end{minipage}
\begin{minipage}{0.4\textwidth}
\begin{flushright} \large
\emph{Advisor:} \\
\fancyname{Dr.~Mark}{Reeves}\footnotemark[2]
\end{flushright}
\end{minipage}
\end{center}

\vspace{.4cm}

%%%%% ABSTRACT %%%%%%
\abstract{\fillin{
Abstract goes here
}}

\vfill
{\noindent {\bf Contributors:}} 
%%%%% CONTRIBUTERS %%%%%%
\fancyname{Zhaowen}{Liu}, \fancyname{Maryam}{Raftari}

\footnotetext[1]{hugadams@gwmail.gwu.edu}
\footnotetext[2]{reevesme@gwu.edu}

\newpage

\tableofcontents

\newpage

\hypertarget{intro}{\section{Introduction}}

\fillin{Explanation of experiment and goals}

\subsection{Hypothesized results}
\fillin{What do you expect to happen?}

\hypertarget{exp}{\section{Experimental}}

\begin{center}
\begin{tabular}{| c | c |}
 \hline
 {\bf Prep. Date} & XX/XX/2014 \\ \hline
 {\bf Prepped by} & Adam Hughes \\ \hline

 {\bf Fiber type} & GIF-625 \\ \hline
 {\bf \# Fibers} & 4  \\ \hline
 {\bf HF Etching time} & N/A \\ \hline
 {\bf Silane type} & TMSDE \\ \hline
 {\bf Annealing time/temp} & 30min 125$^\circ$C  \\ \hline
\end{tabular}
\end{center}

\vspace{.3cm}

\fillin{Any details about the experiment} 

\subsection{Exp: {\color{blue} What went particularly well}}

\begin{itemize}
\item{Nothing to remark}
\end{itemize}

\subsection{Exp: {\color{red} What went particularly poorly}}

\begin{itemize}
\item{Nothing to remark}
\end{itemize}

\hypertarget{sem}{\section{SEM}}

\begin{center}
\begin{tabular}{| c | c |}
 \hline
 {\bf Date of Imaging} & XX/XX/2013 \\ \hline
 {\bf \# Fibers Imaged} & X \\ \hline
 {\bf Microscope Performance} & Good \\ \hline
 {\bf Signs of contamination} & No  \\ \hline
 {\bf Good cleaves} & X/Total  \\ \hline %X out of Y
\end{tabular}
\end{center}
\gap

\subsection{SEM: Notes on each fiber}

% Copy paste below portion once for each image
\gap
\blf{Fiber 1:}
\bit
\item{}
\eit{}


\subsection{SEM: {\color{blue} What went particularly well}}

\begin{itemize}
\item{Nothing to remark}
\end{itemize}

\subsection{SEM: {\color{red} What went particularly poorly}}

\begin{itemize}
\item{Nothing to remark}
\end{itemize}

\subsection{Coverage Analysis \footnotemark[1]}

\hypertarget{coveragetable}{\;}

\texfromfile{\SEM coverage/}{summarytable.tex}{\hspace{2cm}NpSurfaceCounter: python main\_script\_v2.py}
\gap

\fillin{What is surface coverage?  How does it compare to best run and/or related runs?}

\hypertarget{spec}{\section{Spectrometer Protocol}}

\begin{center}
\begin{tabular}{| c | c |}
 \hline
 {\bf Use. Date} & XX/XX/2014 \\ \hline
 {\bf \# Fibers used} & X \\ \hline
 {\bf NP Batch} & 11-18\_B2 30nm diameter\footnotemark[2] \\ \hline
 {\bf Final NP conc.} & .5 stock conc.  \\ \hline
 {\bf Volume of NP solution} & 1mL \\ \hline
\end{tabular}
\end{center}
\gap

\footnotetext[1]{{\bf equiv} refers to the proportion of AuNPs confined to this stucture.  For example, super\_equiv is the estimated percentage of AuNPs confined very large aggregates.}
\footnotetext[2]{30nm refers to batch expected diameter; recent results imply diameter is closer to 22nm.}

\vspace{.3cm}

\fillin{Any particulars about the spectrometer analysis} 

\subsection{Spectrometer: {\color{blue} What went particularly well}}

\begin{itemize}
\item{No handling errors at all.}
\end{itemize}

\subsection{Spectrometer: {\color{red} What went particularly poorly}}

\begin{itemize}
\item{Nothing to remark}
\end{itemize}


\hypertarget{results}{\section{Results}}

\subsection{Summary}

\fillin{Summarize conclusions/results in succinct form}

\subsection{Follow-up}

\
%%%%%%%%%%%%%%%%%%%%%%%%%%%%%%%
%%% SWEAVE TREE .tex %%%%%%%%%%
%%%%%%%%%%%%%%%%%%%%%%%%%%%%%%%
\newpage
\texfromfile{./sweavesections/}{sweavetree.tex}{\hspace{2cm}Reports: python buildsweave.py (see -h for options)}

% IF YOU WANT SWEAVETREE MANUALLY, COMMENT ABOVE, UNCOMMENT AND FILLIN BELOW
%\includepdf[pages=1,pagecommand=\section{Data Analysis}\subsection{Fiber1/NPSAM}]{example.pdf}
%\includepdf[pages=2-,pagecommand={}]{example.pdf}

%%%%%%%%%%%%%%%%%%%%%%%%%%%%%%%
%%% PYUVVIS RUN PARAMS .tex %%%
%%%%%%%%%%%%%%%%%%%%%%%%%%%%%%%
\newpage
\texfromfile{\Analysis}{runparameters.tex}{\hspace{2cm}In root dir: gwuspec Datafiles/ Analysis/ -o -s}


% Appendix
\vspace{.65cm}
\appendix
{\huge \bf \noindent APPENDIX}

% Related (DONT FORGET \_ in filename)
\hypertarget{related}{\section{Related}}
%\begin{itemize}
%\report{October/10_8_NewSilanes_PMTS/Report/10_8_PTMS.pdf}{10\_8: Pure PMTS}
%\end{itemize}
\gap

\section{Acronyms and Terminology} 
\vspace{.1cm}

% XXX: Make the first term the hyperlinked one
\begin{itemize}[align=left]
\define{AuNP}{Gold nanoparticle}

\define{BSA}{Bovine Serum Albumin}

\define{PBS}{Phosphate Buffered Saline- 0.01M pH 7.4 from Sigma Aldrich powder packet.}

\define{TMSDE}{\href{http://www.sigmaaldrich.com/catalog/product/fluka/06666?lang=en&region=US}{\ttfamily(3-Trimethoxysilylpropyl)-diethylenetriamine}}

\define{MPTMS}{\href{http://www.sigmaaldrich.com/catalog/product/aldrich/175617?lang=en&region=US}{\ttfamily(3-Mercaptopropyl)trimethoxysilane}}

\define{PTMS}{\href{http://shop.gelest.com/Product.aspx?catnum=SIP6918.0&Index=0&TotalCount=1}{\ttfamily Propyltrimethoxysilane}}

\define{ETMS}{\href{http://shop.gelest.com/Product.aspx?id=2170}{\ttfamily Ethyltrimethoxysilane}}

\define{DTSSP}{\href{http://www.piercenet.com/product/dsp-lomants-reagent}{\ttfamily Water soluble DSP (Dithiobis[succinimidyl propionate])}}

\end{itemize}

% COVERAGE HISTOGRAMS
\newpage
\section{SEM Images and Histograms}
\texfromfile{\SEM coverage/}{histsummary.tex}{\hspace{2cm}NpSurfaceCounter: python main\_script\_v2.py}

%%%%%%%%%%%%%%%%%%%%%%%%
%%%% END DOCUMENT    %%%
%%%%%%%%%%%%%%%%%%%%%%%%   

\end{document}
